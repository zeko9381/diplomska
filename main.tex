%%%%%%%%%%%%%%%%%%%%%%%%%%%%%%%%%%%%%%%%
% datoteka diploma-FRI-vzorec.tex
%
%POZOR: ta verzija ne producira pdf datoteke v pdf/A formatu!!!
%namenjena je le za nalogo pri Diplomskem seminarju!
%
% vzorčna datoteka za pisanje diplomskega dela v formatu LaTeX
% na UL Fakulteti za računalništvo in informatiko
%
% na osnovi starejših verzij vkup spravil Franc Solina, maj 2021
% prvo verzijo je leta 2010 pripravil Gašper Fijavž
%
% za upravljanje z literaturo ta vezija uporablja BibLaTeX
%
% svetujemo uporabo Overleaf.com - na tej spletni implementaciji LaTeXa ta vzorec zagotovo pravilno deluje
%

\documentclass[a4paper,12pt,openright]{book}
%\documentclass[a4paper, 12pt, openright, draft]{book}  Nalogo preverite tudi z opcijo draft, ki pokaže, katere vrstice so predolge! Pozor, v draft opciji, se slike ne pokažejo!
 
\usepackage[utf8]{inputenc}   % omogoča uporabo slovenskih črk kodiranih v formatu UTF-8
\usepackage[slovenian,english]{babel}    % naloži, med drugim, slovenske delilne vzorce
\usepackage[pdftex]{graphicx}  % omogoča vlaganje slik različnih formatov
\usepackage{fancyhdr}          % poskrbi, na primer, za glave strani
\usepackage{amssymb}           % dodatni matematični simboli
\usepackage{amsmath}           % eqref, npr.
\usepackage[hyphens]{url}
\usepackage{csquotes}
\usepackage[pdftex, colorlinks=true,
						citecolor=black, filecolor=black, 
						linkcolor=black, urlcolor=black,
						pdfproducer={LaTeX}, pdfcreator={LaTeX}]{hyperref}
\usepackage{hyperxmp}
\usepackage{tikz}   % omogoča risanje grafov
\usetikzlibrary{arrows.meta,positioning,fit}

\usepackage{color}
\usepackage{soul}

\usepackage[
backend=biber,
style=numeric,
sorting=nty
]{biblatex}


\addbibresource{main.bib} %Imports bibliography file


%%%%%%%%%%%%%%%%%%%%%%%%%%%%%%%%%%%%%%%%
%	DIPLOMA INFO
%%%%%%%%%%%%%%%%%%%%%%%%%%%%%%%%%%%%%%%%
\newcommand{\ttitle}{Svoboda na dotik: Ali je popolnoma odprtokoden telefon možen?}
\newcommand{\ttitleEn}{Freedom at Your Fingertips: Is a Fully Open-Source Phone Possible?}
\newcommand{\tsubject}{\ttitle}
\newcommand{\tsubjectEn}{\ttitleEn}
\newcommand{\tauthor}{Jan Zajc}
\newcommand{\tkeywords}{odprtokodni telefon, odprtokodna programska oprema, pametni telefoni, mobilni operacijski sistemi, zasebnost}
\newcommand{\tkeywordsEn}{open-source phone, open-source software, smartphones, mobile operating systems, privacy}

%%%%%%%%%%%%%%%%%%%%%%%%%%%%%%%%%%%%%%%%
%	HYPERREF SETUP
%%%%%%%%%%%%%%%%%%%%%%%%%%%%%%%%%%%%%%%%
\hypersetup{pdftitle={\ttitle}}
\hypersetup{pdfsubject=\ttitleEn}
\hypersetup{pdfauthor={\tauthor}}
\hypersetup{pdfkeywords=\tkeywordsEn}

%%%%%%%%%%%%%%%%%%%%%%%%%%%%%%%%%%%%%%%%
% postavitev strani
%%%%%%%%%%%%%%%%%%%%%%%%%%%%%%%%%%%%%%%%  

\addtolength{\marginparwidth}{-20pt} % robovi za tisk
\addtolength{\oddsidemargin}{40pt}
\addtolength{\evensidemargin}{-40pt}

\renewcommand{\baselinestretch}{1.3} % ustrezen razmik med vrsticami
\setlength{\headheight}{15pt}        % potreben prostor na vrhu
\renewcommand{\chaptermark}[1]%
{\markboth{\MakeUppercase{\thechapter.\ #1}}{}} \renewcommand{\sectionmark}[1]%
{\markright{\MakeUppercase{\thesection.\ #1}}} \renewcommand{\headrulewidth}{0.5pt} \renewcommand{\footrulewidth}{0pt}
\fancyhf{}
\fancyhead[LE,RO]{\sl \thepage} 
%\fancyhead[LO]{\sl \rightmark} \fancyhead[RE]{\sl \leftmark}
\fancyhead[RE]{\sc \tauthor}              % dodal Solina
\fancyhead[LO]{\sc Osnutek Diplomske naloge}     % dodal Solina


\newcommand{\BibLaTeX}{{\sc Bib}\LaTeX}
\newcommand{\BibTeX}{{\sc Bib}\TeX}

%%%%%%%%%%%%%%%%%%%%%%%%%%%%%%%%%%%%%%%%
% naslovi
%%%%%%%%%%%%%%%%%%%%%%%%%%%%%%%%%%%%%%%%  

\newcommand{\autfont}{\Large}
\newcommand{\titfont}{\LARGE\bf}
\newcommand{\clearemptydoublepage}{\newpage{\pagestyle{empty}\cleardoublepage}}
\setcounter{tocdepth}{1}	      % globina kazala

%%%%%%%%%%%%%%%%%%%%%%%%%%%%%%%%%%%%%%%%
% konstrukti
%%%%%%%%%%%%%%%%%%%%%%%%%%%%%%%%%%%%%%%%  
\newtheorem{izrek}{Izrek}[chapter]
\newtheorem{trditev}{Trditev}[izrek]
\newenvironment{dokaz}{\emph{Dokaz.}\ }{\hspace{\fill}{$\Box$}}


%%%%%%%%%%%%%%%%%%%%%%%%%%%%%%%%%%%%%%%%%%%%%%%%%%%%%%%%%%%%%%%%%%%%%%%%%%%%%%%
%% PDF-A
%%%%%%%%%%%%%%%%%%%%%%%%%%%%%%%%%%%%%%%%%%%%%%%%%%%%%%%%%%%%%%%%%%%%%%%%%%%%%%%

%%%%%%%%%%%%%%%%%%%%%%%%%%%%%%%%%%%%%%%% 
% define medatata
%%%%%%%%%%%%%%%%%%%%%%%%%%%%%%%%%%%%%%%% 
\def\Title{\ttitle}
\def\Author{\tauthor, jz3623@student.uni-lj.si}
\def\Subject{\ttitleEn}
\def\Keywords{\tkeywordsEn}

%%%%%%%%%%%%%%%%%%%%%%%%%%%%%%%%%%%%%%%% 
% \convertDate converts D:20080419103507+02'00' to 2008-04-19T10:35:07+02:00
%%%%%%%%%%%%%%%%%%%%%%%%%%%%%%%%%%%%%%%% 
\def\convertDate{%
    \getYear
}

{\catcode`\D=12
 \gdef\getYear D:#1#2#3#4{\edef\xYear{#1#2#3#4}\getMonth}
}
\def\getMonth#1#2{\edef\xMonth{#1#2}\getDay}
\def\getDay#1#2{\edef\xDay{#1#2}\getHour}
\def\getHour#1#2{\edef\xHour{#1#2}\getMin}
\def\getMin#1#2{\edef\xMin{#1#2}\getSec}
\def\getSec#1#2{\edef\xSec{#1#2}\getTZh}
\def\getTZh +#1#2{\edef\xTZh{#1#2}\getTZm}
\def\getTZm '#1#2'{%
    \edef\xTZm{#1#2}%
    \edef\convDate{\xYear-\xMonth-\xDay T\xHour:\xMin:\xSec+\xTZh:\xTZm}%
}

%\expandafter\convertDate\pdfcreationdate 

%%%%%%%%%%%%%%%%%%%%%%%%%%%%%%%%%%%%%%%%
% get pdftex version string
%%%%%%%%%%%%%%%%%%%%%%%%%%%%%%%%%%%%%%%% 
\newcount\countA
\countA=\pdftexversion
\advance \countA by -100
\def\pdftexVersionStr{pdfTeX-1.\the\countA.\pdftexrevision}


%%%%%%%%%%%%%%%%%%%%%%%%%%%%%%%%%%%%%%%%
% XMP data
%%%%%%%%%%%%%%%%%%%%%%%%%%%%%%%%%%%%%%%%  
\usepackage{xmpincl}
%\includexmp{pdfa-1b}

%%%%%%%%%%%%%%%%%%%%%%%%%%%%%%%%%%%%%%%%
% pdfInfo
%%%%%%%%%%%%%%%%%%%%%%%%%%%%%%%%%%%%%%%%  
\pdfinfo{%
    /Title    (\ttitle)
    /Author   (\tauthor)
    /Subject  (\ttitleEn)
    /Keywords (\tkeywordsEn)
    /ModDate  (\pdfcreationdate)
    /Trapped  /False
}

%%%%%%%%%%%%%%%%%%%%%%%%%%%%%%%%%%%%%%%%
% znaki za copyright stran
%%%%%%%%%%%%%%%%%%%%%%%%%%%%%%%%%%%%%%%%  

\newcommand{\CcImageCc}[1]{%
	\includegraphics[scale=#1]{cc_cc_30.pdf}%
}
\newcommand{\CcImageBy}[1]{%
	\includegraphics[scale=#1]{cc_by_30.pdf}%
}
\newcommand{\CcImageSa}[1]{%
	\includegraphics[scale=#1]{cc_sa_30.pdf}%
}

%%%%%%%%%%%%%%%%%%%%%%%%%%%%%%%%%%%%%%%%%%%%%%%%%%%%%%%%%%%%%%%%%%%%%%%%%%%%%%%
%%%%%%%%%%%%%%%%%%%%%%%%%%%%%%%%%%%%%%%%%%%%%%%%%%%%%%%%%%%%%%%%%%%%%%%%%%%%%%%

\begin{document}
\selectlanguage{slovene}
\frontmatter
\setcounter{page}{1} %
\renewcommand{\thepage}{}       % preprečimo težave s številkami strani v kazalu

%%%%%%%%%%%%%%%%%%%%%%%%%%%%%%%%%%%%%%%%
%naslovnica
 \thispagestyle{empty}%
   \begin{center}
    {\large\sc Univerza v Ljubljani\\%
%      Fakulteta za elektrotehniko\\% za študijski program Multimedija
%      Fakulteta za upravo\\% za študijski program Upravna informatika
      Fakulteta za računalništvo in informatiko\\%
%      Fakulteta za matematiko in fiziko\\% za študijski program Računalništvo in matematika
     }
    \vskip 10em%
    {\autfont \tauthor\par}%
    {\titfont \ttitle \par}%
    {\vskip 3em \textsc{OSNUTEK DIPLOMSKE NALOGE\\[5mm]         % dodal Solina za ostale študijske programe
    VISOKOŠOLSKI STROKOVNI ŠTUDIJSKI PROGRAM\\ PRVE STOPNJE\\ RAČUNALNIŠTVO IN INFORMATIKA}\par}%
%     UNIVERZITETNI  ŠTUDIJSKI PROGRAM\\ PRVE STOPNJE\\ RAČUNALNIŠTVO IN INFORMATIKA}\par}%
%    INTERDISCIPLINARNI UNIVERZITETNI\\ ŠTUDIJSKI PROGRAM PRVE STOPNJE\\ MULTIMEDIJA}\par}%
%    INTERDISCIPLINARNI UNIVERZITETNI\\ ŠTUDIJSKI PROGRAM PRVE STOPNJE\\ UPRAVNA INFORMATIKA}\par}%
%    INTERDISCIPLINARNI UNIVERZITETNI\\ ŠTUDIJSKI PROGRAM PRVE STOPNJE\\ RAČUNALNIŠTVO IN MATEMATIKA}\par}%
    \vfill\null%
% izberite pravi habilitacijski naziv mentorja!
    {\large \textsc{Mentor}: viš. pred. dr. David Jelenc\par}%
%   {\large \textsc{Somentor}:  viš. pred./doc./izr. prof./prof. dr.  Martin Krpan \par}%
    {\vskip 2em \large Ljubljana, \the\year \par}%
\end{center}
% prazna stran
%\clearemptydoublepage      
% izjava o licencah itd. se izpiše na hrbtni strani naslovnice

%%%%%%%%%%%%%%%%%%%%%%%%%%%%%%%%%%%%%%%%
%copyright stran
%%%%%%%%%%%%%%%%%%%%%%%%%%%%%%%%%%%%%%%%
\newpage
\thispagestyle{empty}

\vspace*{5cm}
{\small \noindent
To delo je ponujeno pod licenco \textit{Creative Commons Priznanje avtorstva-Deljenje pod enakimi pogoji 2.5 Slovenija} (ali novej\v so razli\v cico).
To pomeni, da se tako besedilo, slike, grafi in druge sestavine dela kot tudi rezultati diplomskega dela lahko prosto distribuirajo,
reproducirajo, uporabljajo, priobčujejo javnosti in predelujejo, pod pogojem, da se jasno in vidno navede avtorja in naslov tega
dela in da se v primeru spremembe, preoblikovanja ali uporabe tega dela v svojem delu, lahko distribuira predelava le pod
licenco, ki je enaka tej.
Podrobnosti licence so dostopne na spletni strani \href{http://creativecommons.si}{creativecommons.si} ali na Inštitutu za
intelektualno lastnino, Streliška 1, 1000 Ljubljana.

\vspace*{1cm}
\begin{center}% 0.66 / 0.89 = 0.741573033707865
\CcImageCc{0.741573033707865}\hspace*{1ex}\CcImageBy{1}\hspace*{1ex}\CcImageSa{1}%
\end{center}
}

\vspace*{1cm}
{\small \noindent
Izvorna koda diplomskega dela, njeni rezultati in v ta namen razvita programska oprema je ponujena pod licenco GNU General Public License,
različica 3 (ali novejša). To pomeni, da se lahko prosto distribuira in/ali predeluje pod njenimi pogoji.
Podrobnosti licence so dostopne na spletni strani \url{http://www.gnu.org/licenses/}.
}

\vfill
\begin{center} 
\ \\ \vfill
{\em
Besedilo je oblikovano z urejevalnikom besedil \LaTeX.}
\end{center}

% prazna stran
\clearemptydoublepage

%%%%%%%%%%%%%%%%%%%%%%%%%%%%%%%%%%%%%%%%
% stran 3 med uvodnimi listi
\thispagestyle{empty}
\
\vfill

\bigskip
\noindent\textbf{Kandidat:} Jan Zajc\\
\noindent\textbf{Naslov:} Svoboda na dotik: Ali je popolnoma odprtokoden telefon možen?\\
% vstavite ustrezen naziv študijskega programa!
\noindent\textbf{Vrsta naloge:} Diplomska naloga na visokošolskem strokovnem programu prve stopnje Računalništvo in informatika \\
% izberite pravi habilitacijski naziv mentorja!
\noindent\textbf{Mentor:} viš. pred. dr. David Jelenc \\
%\noindent\textbf{Somentor:} isto kot za mentorja

\bigskip
\noindent\textbf{Opis:}\\
Opis diplome

\bigskip
\noindent\textbf{Title:} Freedom at Your Fingertips: Is a Fully Open-Source Phone Possible?

\bigskip
\noindent\textbf{Description:}\\
Opis diplome v angleščini

\vfill



\vspace{2cm}

% prazna stran
\clearemptydoublepage

% zahvala
%\thispagestyle{empty}\mbox{}\vfill\null\it%
%\noindent
%Na tem mestu zapišite, komu se zahvaljujete za pomoč pri izdelavi diplomske naloge oziroma pri vašem študiju nasploh. Pazite, da ne boste koga pozabili. Utegnil vam bo zameriti. Temu se da izogniti tako, da celotno zahvalo izpustite.
%\rm\normalfont

% prazna stran
\clearemptydoublepage

%%%%%%%%%%%%%%%%%%%%%%%%%%%%%%%%%%%%%%%%
% posvetilo, če sama zahvala ne zadošča :-)
%\thispagestyle{empty}\mbox{}{\vskip0.20\textheight}\mbox{}\hfill\begin{minipage}{0.55\textwidth}%
%Svoji dragi Alenčici.
%\normalfont\end{minipage}

% prazna stran
%\clearemptydoublepage


%%%%%%%%%%%%%%%%%%%%%%%%%%%%%%%%%%%%%%%%
% kazalo
\pagestyle{empty}
\def\thepage{}% preprečimo težave s številkami strani v kazalu
\tableofcontents{}


% Tabela s kraticami


% prazna stran
\clearemptydoublepage

%%%%%%%%%%%%%%%%%%%%%%%%%%%%%%%%%%%%%%%%
% povzetek
\addcontentsline{toc}{chapter}{Povzetek}
\chapter*{Povzetek}

\noindent\textbf{Naslov:} \ttitle
\bigskip

\noindent\textbf{Avtor:} \tauthor
\bigskip

%\noindent\textbf{Povzetek:} 
\noindent placeholder

\bigskip

\noindent\textbf{Ključne besede:} \tkeywords.
% prazna stran
\clearemptydoublepage

%%%%%%%%%%%%%%%%%%%%%%%%%%%%%%%%%%%%%%%%
% abstract
\selectlanguage{english}
\addcontentsline{toc}{chapter}{Abstract}
\chapter*{Abstract}

\noindent\textbf{Title:} \ttitleEn
\bigskip

\noindent\textbf{Author:} \tauthor
\bigskip

%\noindent\textbf{Abstract:} 
\noindent placeholder
\bigskip

\noindent\textbf{Keywords:} \tkeywordsEn.
\selectlanguage{slovene}
% prazna stran
\clearemptydoublepage

%%%%%%%%%%%%%%%%%%%%%%%%%%%%%%%%%%%%%%%%
\mainmatter
\setcounter{page}{1}
\pagestyle{fancy}

\chapter{Uvod}
\label{chap:uvod}
placeholder

\chapter{Pregled področja: mobilni operacijski sistemi in odprtokodne alternative}
\label{chap:pregled_podrocja}

Mobilni telefoni so se iz komunikacijskih naprav razvili v splošnonamenska računalniška orodja, ki jih uporabljamo za delo, zabavo, navigacijo, avtentikacijo in komunikacijo.
V praksi to pomeni, da je operacijski sistem pametnega telefona postal ena izmed ključnih komponent digitalnega življenja uporabnika.
Trenutno stanje na trgu je močno centralizirano.
Mobilni ekosistem je v veliki meri razdeljen med Android in iOS, kar ima neposredne posledice za svobodo uporabnika, zasebnost in življenjsko dobo naprav.
V tem poglavju najprej opišemo ključne lastnosti obeh prevladujočih platform, nato pa pregledamo glavne odprtokodne pristope in njihove omejitve.

\section{Terminologija}
\label{sec:terminologija}

V poglavju uporabljamo nekaj izrazov, ki se pri mobilnih sistemih pogosto pojavljajo v angleščini ali pa imajo v slovenščini več možnih prevodov. Da ne bi pri vsaki sekciji znova razlagali istih pojmov, spodaj na kratko zberemo ključne izraze in kratice. V nadaljevanju jih uporabljamo dosledno v istem pomenu.

\begin{description}
  \item[Android, iOS] Dve prevladujoči mobilni platformi. Android je ekosistem naprav različnih proizvajalcev, iOS pa Applov operacijski sistem za iPhone.

  \item[Applova trgovina z aplikacijami (\textit{App Store})] Uradni kanal za distribucijo aplikacij na iOS.

  \item[Androidov odprtokodni projekt (\textit{Android Open Source Project}, AOSP)] Odprtokodna osnova Androida, na kateri temeljijo številne distribucije in prilagoditve.

  \item[Izdelovalec originalne opreme (OEM)] Proizvajalec končne naprave (npr. OnePlus, Samsung), ki tipično prilagodi sistem in programsko opremo za svojo strojno opremo.

  \item[Sistem na čipu (SoC)] Glavni čip v telefonu, ki običajno vključuje CPU, GPU, komunikacijske podsisteme in druge krmilnike (npr. Qualcomm Snapdragon).

  \item[Gonilnik] Programska komponenta, ki operacijskemu sistemu omogoča uporabo strojne opreme (angl. \textit{driver}).

  \item[Strojna programska oprema (\textit{firmware})] Programi, ki tečejo neposredno na strojnih komponentah (npr. kamera, Wi-Fi/BT, modem) in so pogosto lastniški.

  \item[Sklad dobavitelja (\textit{vendor stack})] Strojno-specifičen del programskega sklada, ki ga zagotovi dobavitelj čipovja ali proizvajalec naprave (gonilniki, \textit{firmware}, vmesniki).

  \item[Googlove mobilne storitve (\textit{Google Mobile Services}, GMS)] Nabor lastniških Googlovih komponent na Androidu (npr. trgovina aplikacij, storitve za obvestila, API-ji). Pomemben del predstavlja \textit{GMS Core} oziroma \textit{Google Play Services}.

  \item[Programski vmesnik (API)] Vmesnik, prek katerega aplikacije ali sistemske komponente uporabljajo funkcionalnosti drugih komponent (angl. \textit{Application Programming Interface}).

  \item[Potisna obvestila (\textit{push notifications})] Mehanizem, kjer strežnik sproži obvestilo aplikaciji na napravi (pogosto prek posrednih storitev).

  \item[Preverjanje integritete (SafetyNet/Play Integrity)] Mehanizmi, s katerimi ponudnik platforme preverja lastnosti naprave in okolja izvajanja (npr. certificiranost, nepooblaščene spremembe). SafetyNet je starejši pristop, Play Integrity pa novejši.

  \item[Upravljanje digitalnih pravic (\textit{Digital Rights Management}, DRM)] Tehnike za nadzor dostopa do zaščitenih vsebin (npr. video). V kontekstu Androida je pogost primer \textit{Widevine}, ki uporablja različne varnostne ravni (npr. L1, L3).

  \item[Jedro] Osrednji del operacijskega sistema (angl. \textit{kernel}). V mobilnem svetu je pogosto pomembno, ali naprava uporablja jedro iz glavne razvojne veje.

  \item[Uporabniški prostor (\textit{userspace})] Del sistema, kjer teče večina programov in storitev nad jedrom (npr. lupina, paketni upravljalnik, grafični vmesnik).

  \item[Jedro iz glavne razvojne veje (\textit{mainline kernel})] Jedro Linuxa, ki je del uradne (glavne) razvojne veje, ne pa posebej prilagojena različica proizvajalca.

  \item[ROM] V kontekstu Androida izraz uporabljamo za sistemsko sliko oziroma distribucijo operacijskega sistema (npr. \textit{stock ROM} za tovarniški sistem, \textit{custom ROM} za alternativni sistem). Ne gre za pomnilniški čip tipa ROM.

  \item[Skrbniški dostop (\textit{root})] Možnost izvajanja ukazov z najvišjimi privilegiji v sistemu.

  \item[Obnovitveno okolje (\textit{recovery})] Posebno zagonsko okolje za vzdrževanje naprave (namestitve, posodobitve, brisanje podatkov ipd.).

  \item[Prenos (\textit{porting})] Postopek prilagoditve operacijskega sistema ali distribucije na določeno strojno platformo oziroma napravo.

  \item[Osnovnopasovni podsistem (\textit{baseband})] Komunikacijski podsistem (modem) za mobilna omrežja, pogosto ločen od glavnega procesorja in z lastno strojno programsko opremo.

  \item[Konvergenca (\textit{convergence})] Koncept, kjer ista naprava (telefon) v določenih načinih uporabe deluje tudi kot namizni računalnik.

  \item[Vmesni sloji (libhybris, Halium)] Pristopi, ki omogočajo uporabo Androidovih strojno-specifičnih komponent (npr. knjižnic/gonilnikov) v okolju GNU/Linux na telefonih.
\end{description}

\section{Duopol Android/iOS in posledice za uporabnike}
\label{sec:duopol}

Android in iOS predstavljata praktično celoten trg mobilnih operacijskih sistemov.
Po podatkih StatCounter ima Android globalno večinski delež, iOS pa večino preostalega~\cite{statcounter_mobileos_2025}.
Podobno sliko kažejo tudi tržne analize prodajnih deležev~\cite{counterpoint_os_share_2025}.
Ne glede na natančne odstotke je ključna ugotovitev, da sta v praksi relevantni skoraj izključno ti dve platformi.

Takšna koncentracija moči pomeni, da sta Google in Apple posrednika med uporabnikom in napravo.
Določata, kako se namešča programska oprema, kakšna so pravila za aplikacije, kakšni so varnostni mehanizmi in katere funkcionalnosti so sploh dostopne.
Čeprav sta varnost in zanesljivost pomembni, tak model pogosto vodi v položaj, kjer se uporabnikove pravice (npr. nadzor nad sistemom, možnost prilagoditev, izbira storitev) podredijo poslovnim interesom ekosistema.

\section{iOS: zaprt ekosistem in nadzor nad distribucijo aplikacij}
\label{sec:ios}

iOS je zasnovan kot strogo nadzorovan sistem.
Uporabnik praviloma namešča aplikacije prek uradne Applove trgovine z aplikacijami (angl. \textit{App Store}), sistem pa je zgrajen okoli modela podpisovanja kode in preverjanja integritete~\cite{apple_app_code_signing,apple_app_review_guidelines}.
Tak pristop ima prednosti, kot so enotna uporabniška izkušnja, relativno predvidljiv varnostni model in dolga programska podpora za naprave.
Hkrati pa pomeni tudi, da ima uporabnik omejen vpliv na to, kaj se lahko poganja na napravi in kakšne posege v sistem je sploh možno izvesti.

Z vidika diplomske naloge je pomembno predvsem to, da iOS kot platforma ne omogoča resnične možnosti ``popolnoma odprtokodnega telefona''.
Uporabnik niti teoretično nima možnosti zamenjati ključnih komponent sistema z odprtokodnimi alternativami.
iOS je zato dober kontrast, saj pokaže, kako je videti maksimalno centraliziran mobilni ekosistem, kjer je nadzor nad napravo primarno v rokah proizvajalca~\cite{apple_app_review_guidelines}.

\section{Android: odprtokodna osnova in resničnost zaprtih komponent}
\label{sec:android}

Android se pogosto predstavlja kot odprtokoden sistem, saj temelji na Androidovem odprtokodnem projektu (angl. \textit{Android Open Source Project}, AOSP)~\cite{aosp_get_started_2025}.
To je pomembna razlika v primerjavi z iOS, ker AOSP omogoča javno dostopno izvorno kodo za velik del platforme in teoretično dopušča razvoj alternativnih distribucij.

V praksi pa tipična naprava z Androidom ni enaka AOSP.
Večina naprav vsebuje:
\begin{itemize}
  \item prilagoditve izdelovalca originalne opreme (OEM), ki pogosto vključujejo dodatne sistemske aplikacije;
  \item zaprtokodne gonilnike in strojno programsko opremo (angl. \textit{firmware}) za posamezne komponente, ki so del sklada dobavitelja (angl. \textit{vendor stack})~\cite{android_treble_vendor_system_split,android_hidl_versioning_treble};
  \item Googlove mobilne storitve (angl. \textit{Google Mobile Services}, GMS), ki niso del AOSP in so na voljo le prek licence; pomemben del teh storitev predstavlja GMS Core oziroma Google Play Services~\cite{gms_not_part_of_aosp_license,gms_core_proprietary_layer}.
\end{itemize}

Posledica je, da je Android kot ekosistem hibrid.
Odprtokodna osnova se v praksi nadgradi z več lastniškimi plastmi.
To se najbolj pozna pri strojni podpori.
Gonilniki za grafični pospeševalnik, kamero, mobilni modem in druge ključne komponente so pogosto zaprti ter vezani na konkretno jedro in različico sistema~\cite{android_treble_vendor_system_split,android_hidl_versioning_treble}.
Zaradi tega posodobitve niso zgolj vprašanje ``nove različice Androida'', temveč tudi vprašanje združljivosti celotnega sklada gonilnikov in strojne programske opreme.
Del problema Android poskuša nasloviti z modularizacijo sistemskih komponent (projekt \textit{Mainline}), kjer je del posodobitev mogoče dostaviti ločeno od celotnega sistema~\cite{android_mainline_modular_system}.

Ena izmed posledic takšne arhitekture je tudi fragmentacija.
Različni proizvajalci vzdržujejo svoje različice sistema in jedra, kar poveča stroške vzdrževanja in lahko negativno vpliva na hitrost ter trajanje zagotavljanja posodobitev~\cite{acar_android_security_updates_2024,android_devblog_treble_plus_one_2020}.
To je eden izmed razlogov, zakaj naprave pogosto dobijo le omejeno število večjih posodobitev, čeprav je strojna oprema še povsem zmogljiva.
Kot minimalen referenčni okvir lahko uporabimo tudi zahteve programa ``Android Enterprise Recommended'', ki za vključene naprave predvideva podporo trenutni izdaji in vsaj eni večji nadgradnji operacijskega sistema~\cite{android_enterprise_recommended_requirements}.

\section{Akterji mobilnega ekosistema}
\label{sec:akterji}

Za razumevanje omejitev odprtokodnih mobilnih sistemov je ključno upoštevati, da telefon ni samo operacijski sistem in aplikacije.
Pri tipični napravi sodeluje več akterjev:
\begin{itemize}
  \item \textbf{Proizvajalec operacijskega sistema oziroma ekosistema} (Apple ali Google, posredno tudi skupnost AOSP);
  \item \textbf{Izdelovalec originalne opreme (OEM)} (npr. OnePlus), ki prilagodi sistem;
  \item \textbf{Proizvajalec sistema na čipu (SoC)} (npr. Qualcomm), ki zagotovi gonilnike in strojno programsko opremo (angl. \textit{firmware});
  \item \textbf{Operaterji} (v določenih državah tudi certificiranje in omejitve);
  \item \textbf{Razvijalci aplikacij}, ki se pogosto zanašajo na lastniške programske vmesnike (API) in storitve (npr. potisna obvestila, SafetyNet/Play Integrity)~\cite{safetynet_deprecated_replaced,play_integrity_overview}.
\end{itemize}

Poseben primer je \textbf{osnovnopasovni podsistem (angl. \textit{baseband})}.
Ta je pogosto ločen procesor ali podsistem s svojo strojno programsko opremo, ki je praviloma lastniška~\cite{hernandez_firmwire_ndss2022,davis_cellular_baseband_security_2012}.
Zaradi varnosti in regulative je razumljivo, zakaj je ta del izoliran, vendar z vidika odprtokodnosti predstavlja trdno omejitev.
Tudi če zamenjamo operacijski sistem in aplikacije, ostaja osnovnopasovni podsistem ključen zaprt del naprave.

\section{Alternativni Android ROM-i}
\label{sec:custom_rom}

Alternativni ROM-i za Android so pogost način, kako predvsem tehnično bolj podkovani uporabniki poskušajo povečati nadzor nad napravo.
Najbolj poznan primer je LineageOS, poleg tega pa obstajajo še projekti, ki se posebej osredotočajo na zasebnost ali zmanjšanje odvisnosti od Googlovih storitev (npr. /e/OS, GrapheneOS)~\cite{lineageos_home,eos_about,grapheneos_about}.

V primerjavi s tovarniškimi ROM-i alternative pogosto ponujajo:
\begin{itemize}
  \item manj prednaložene odvečne programske opreme (angl. \textit{bloatware});
  \item bolj transparentne nastavitve zasebnosti;
  \item več možnosti prilagajanja (npr. skrbniški dostop, angl. \textit{root}, drugačno obnovitveno okolje, angl. \textit{recovery}, ter večji nadzor nad sistemskimi aplikacijami);
  \item v nekaterih primerih daljšo podporo za starejše naprave (odvisno od skupnosti in vzdrževanja)~\cite{bradley_sustainable_security_2023}.
\end{itemize}

Ključna omejitev alternativnih ROM-ov je, da so še vedno vezani na isti problem kot tovarniški Android.
Odvisni so od zaprtokodnih gonilnikov in strojne programske opreme (angl. \textit{firmware}), ki jih zagotavlja proizvajalec strojne opreme.
Poleg tega se v zadnjih letih povečuje odvisnost aplikacij od lastniških storitev in mehanizmov preverjanja integritete (npr. Play Integrity), kar lahko na sistemih brez Googlovih storitev povzroča težave pri aplikacijah z višjimi varnostnimi zahtevami (npr. plačila, bančne aplikacije)~\cite{play_integrity_overview,google_play_integrity_help}.
Pri zaščiti vsebin se uporabljajo mehanizmi upravljanja digitalnih pravic (angl. \textit{Digital Rights Management}, DRM), kot je Widevine, kjer lahko razlike v varnostnih ravneh vplivajo na razpoložljive funkcionalnosti oziroma kakovost predvajanja~\cite{widevine_overview_2024,widevine_help_security_levels,chrome_media_updates_chrome62_2017,bitmovin_widevine_levels}.

\section{Mobilni Linux: postmarketOS, Ubuntu Touch in sorodni projekti}
\label{sec:mobile_linux}

Mobilni Linux poskuša na telefon prinesti bolj klasičen model računalniškega sistema.
V ospredju so standarden uporabniški prostor Linuxa (angl. \textit{userspace}), paketni upravljalnik, terminal in večja svoboda prilagajanja.
Med bolj aktivnimi projekti sta postmarketOS in Ubuntu Touch.

Projekt postmarketOS gradi na distribuciji Alpine Linux in ima dolgoročno usmeritev v vzdržljivost ter ponovno uporabnost naprav~\cite{postmarketos_about}.
S tem poskuša zmanjšati odvisnost od specifičnih različic Androida in cilja na to, da bi naprave lahko uporabljale jedro iz glavne razvojne veje (angl. \textit{mainline kernel}), kjer je to izvedljivo.

Ubuntu Touch razvija skupnost UBports in poskuša ponuditi uporabniku prijazen mobilni sistem z lastnim grafičnim vmesnikom.
Pomemben del zasnove je koncept konvergence, kjer lahko ista naprava deluje kot telefon in kot namizni računalnik~\cite{ubports_ubuntu_touch,ubports_convergence}.
Cilj je ponuditi alternativo Androidu, ki ni vezana na Googlov ekosistem in kjer je večina programske opreme odprtokodna.

Kljub obetom je resničnost mobilnega Linuxa še vedno močno odvisna od strojne podpore.
Pri številnih napravah se za delovanje še vedno uporabljajo obhodi, kot so vmesni sloji za uporabo Android gonilnikov (npr. libhybris, Halium), kar pomeni kompromis med odprtostjo in praktično uporabnostjo~\cite{halium_docs_intro,libhybris_readme,sailfish_hadk_libhybris}.
To je tipičen primer konflikta.
Ideal je ``čist'' Linux programski sklad, praksa pa pogosto zahteva uporabo obstoječih lastniških komponent.

\section{Namenske Linux naprave}
\label{sec:linux_phones}

Poleg prenosov na obstoječe telefone Android (angl. \textit{porting}) obstajajo tudi namenske naprave, ki ciljajo na uporabnike, ki želijo odprtokodne alternative (npr. PinePhone, Librem~5)~\cite{pine64_pinephone_device,pine64_wiki_pinephone,purism_librem5_product,purism_librem5_docs}.
Te naprave pogosto poskušajo izboljšati stanje tako, da izberejo strojno opremo z boljšo podporo v jedru Linuxa ali z bolj odprto dokumentacijo.

Prednost namenskih Linux telefonov je, da je programski sklad pogosto bolj konsistenten (jedro + uporabniški prostor + uporabniški vmesnik) in da skupnost razvoj usmeri na manjše število naprav.
Slabost pa je, da se takšne naprave pogosto ne morejo kosati z najboljšimi telefoni z Androidom ali iOS glede zmogljivosti, kakovosti kamere in porabe energije.
Zato ostajajo predvsem v niši, kar vpliva tudi na velikost skupnosti in hitrost razvoja.

\section{Povzetek}
\label{sec:povzetek_pregleda}

Pregled področja pokaže, da mobilni ekosistem trenutno obvladuje duopol Android/iOS, kjer je uporabniška svoboda v veliki meri omejena z zaprtimi ekosistemi, lastniškimi komponentami in poslovnimi modeli.
Android sicer ponuja odprtokodno osnovo (AOSP), vendar tipične naprave vsebujejo veliko zaprtih slojev, kar oteži dolgotrajno vzdrževanje in razvoj alternativ.
Alternativni ROM-i za Android delno izboljšajo nadzor nad sistemom, vendar ostajajo odvisni od istih zaprtih gonilnikov in strojne programske opreme.
Projekti mobilnega Linuxa ponujajo bolj radikalno alternativo, vendar je njihova praktična uporabnost še vedno močno omejena s strojno podporo in ekosistemom aplikacij.
To neposredno motivira nadaljnja poglavja, kjer bodo najprej opredeljeni kriteriji ``popolnoma odprtokodnega telefona'', nato pa bodo analizirane obstoječe rešitve in empirično preizkušene.


\chapter{Teoretična izhodišča}
\label{chap:teoreticna_izhodisca}

V prejšnjem poglavju smo pokazali, da je tudi pri sistemih z odprtokodno osnovo velik del mobilnega ekosistema odvisen od zaprtih komponent.
Da lahko kasneje ocenimo, v kolikšni meri je \enquote{popolnoma odprtokoden telefon} v praksi izvedljiv, moramo najprej razjasniti osnovne pojme, sloje sistema in tipične omejitve.
V tem poglavju opredelimo razliko med odprtokodno in prosto programsko opremo, na kratko predstavimo arhitekturo mobilnih sistemov ter izpostavimo vlogo gonilnikov in strojne programske opreme.
Nato povežemo odprtost z varnostjo in zasebnostjo, pri čemer obravnavamo tudi tveganja skritih funkcionalnosti oziroma stranskih vrat (v literaturi pogosto označenih kot angl.~\textit{backdoor}).

\section{Odprtokodna in prosta programska oprema}
\label{sec:teorija_foss}

Pojma odprtokodna programska oprema (angl.~\textit{open source software}) in prosta programska oprema (angl.~\textit{free software}) se pogosto uporabljata kot sopomenki, vendar izhajata iz različnih poudarkov.
Open Source Initiative (OSI) odprtokodnost definira s kriteriji, ki med drugim zahtevajo dostop do izvorne kode, možnost sprememb in redistribucije ter nediskriminatorne licence~\cite{osi_osd}.
Free Software Foundation (FSF) pa v ospredje postavlja uporabniške svoboščine: program je prost, če omogoča (1) poganjanje za poljuben namen, (2) preučevanje delovanja, (3) prilagajanje ter (4) deljenje kopij, tudi spremenjenih~\cite{fsf_free_software}.
V nadaljevanju zato uporabljamo tudi širši izraz prosta in odprtokodna programska oprema (angl.~\textit{Free and Open Source Software}, FOSS), kadar želimo zajeti oba vidika.

Pri ocenjevanju \enquote{odprtosti telefona} je pomembno, da ločimo licenco od samega dejstva, da je koda javno dostopna.
V praksi se pojavljajo tudi projekti, kjer je izvorna koda objavljena, vendar licenca ne dopušča redistribucije ali sprememb; za tak model se pogosto uporablja izraz dostopna izvorna koda (angl.~\textit{source-available}).
Za kriterije v tej diplomski nalogi zato ne zadošča zgolj javna objava kode, ampak zahtevamo tudi pravice, ki jih določajo licence, ter možnost dejanske uporabe spremenjenih različic (npr.~namestitev na napravo).

\subsection{Licence in posledice za mobilne sisteme}
\label{subsec:teorija_licence}

Licence določajo, kaj lahko uporabnik in razvijalec s kodo počneta.
V grobem jih lahko razdelimo na:
\begin{itemize}
  \item permisivne licence (npr.~MIT, BSD, Apache~2.0), ki omogočajo tudi vključitev kode v zaprt izdelek, in
  \item licence z zahtevami deljenja izboljšav (angl.~\textit{copyleft}, npr.~GPL), ki ob distribuciji pod določenimi pogoji zahtevajo, da so tudi izpeljana dela na voljo pod enako (ali združljivo) licenco.
\end{itemize}

Za telefone je posebej pomembno jedro Linux, ki je pod licenco GPLv2.
To proizvajalcem pri distribuciji naprav nalaga obveznosti glede objave izvorne kode sprememb jedra, vendar še ne zagotavlja, da je odprt celoten sistem.
V praksi lahko ostanejo zaprti gonilniki, strojna programska oprema in velik del uporabniškega prostora (npr.~sistemske aplikacije ali dobaviteljske knjižnice).

\section{Arhitektura sistema: od zagona do uporabniškega prostora}
\label{sec:teorija_arhitektura}

Pametni telefon je sestavljen iz več podsistemov.
Poleg glavnega procesorja (SoC) ima naprava pogosto ločene mikrokrmilnike ali koprocesorje za modem, Wi-Fi/Bluetooth, kamero in senzorje, pogosto pa tudi varnostne podsisteme, kot je zaupanja vredno izvajalno okolje (angl.~\textit{Trusted Execution Environment}, TEE).
Ti deli praviloma uporabljajo lastno strojno programsko opremo in so pomemben del vprašanja zaupanja, ker so za uporabnika običajno netransparentni.

Na nivoju glavnega sistema ločimo:
\begin{itemize}
  \item zagonsko verigo, ki določa, kaj se sploh sme zagnati (npr.~zagonski nalagalnik (angl.~\textit{bootloader}) in preverjen zagon (angl.~\textit{verified boot})),
  \item jedro (angl.~\textit{kernel}) z gonilniki in
  \item uporabniški prostor (angl.~\textit{userspace}), kjer tečejo sistemske storitve, knjižnice in aplikacije.
\end{itemize}

Slika~\ref{fig:arhitektura_telefona} prikazuje tipično razmejitev med splošnimi sloji sistema in zaprtimi dobaviteljskimi komponentami, ki pogosto predstavljajo ključno omejitev pri \enquote{popolnoma odprtokodnem} telefonu.

\begin{figure}[!t]
  \centering
  \begin{tikzpicture}[
    font=\small,
    box/.style={draw, rounded corners, align=center, text width=5.2cm, minimum height=9mm},
    open/.style={box, fill=black!3},
    closed/.style={box, dashed, fill=black!10},
    arrow/.style={-Latex, thick},
    node distance=3.2mm and 10mm
  ]
    % Levo: "odprti" sloji
    \node[open] (boot) {zagonska veriga\\(\textit{boot ROM}, \textit{bootloader}, \textit{verified boot})};
    \node[open, below=of boot] (kernel) {jedro\\(gonilniki, scheduler)};
    \node[open, below=of kernel] (user) {uporabniški prostor\\(storitve, knjižnice, aplikacije)};

    \draw[arrow] (boot) -- (kernel);
    \draw[arrow] (kernel) -- (user);

    \node[draw, rounded corners, fit=(boot)(user), inner sep=6mm,
          label={[font=\small]above:glavni sistem (SoC)}] (soc) {};

    % Desno: zaprt del (en stolpec)
    \node[closed, right=of kernel] (vendor) {dobaviteljski sklad\\(\texttt{vendor}, HAL, knjižnice)};
    \node[closed, below=of vendor] (fw) {strojna programska oprema\\(\textit{firmware}, \textit{binary blobs})};
    \node[closed, below=of fw, text width=5.2cm, align=left] (subsys) {podsistemi (pogosto zaprti):\\
      -- osnovnopasovni modem (\textit{baseband})\\
      -- Wi-Fi / Bluetooth\\
      -- TEE / varnostni element};

    \draw[arrow] (kernel.east) -- node[above, font=\scriptsize]{vmesniki} (vendor.west);
    \draw[arrow] (vendor) -- (fw);
    \draw[arrow] (fw) -- (subsys);

    \node[draw, rounded corners, dashed, fit=(vendor)(subsys), inner sep=6mm,
          label={[font=\small]above:tipično zaprt del}] (closedfit) {};

    % Meja zaupanja (črtkana navpičnica med stolpcema)
    \draw[densely dotted, thick]
      ([xshift=-6mm]vendor.west) -- ([xshift=-6mm]subsys.south);
    \node[font=\scriptsize, align=left] at ([xshift=-3mm,yshift=2mm]vendor.west) {meja\\zaupanja};

  \end{tikzpicture}
  \caption{Poenostavljena arhitektura telefona z vidika slojev sistema in tipičnih zaprtih komponent.}
  \label{fig:arhitektura_telefona}
\end{figure}


\subsection{Linux: glavna razvojna veja jedra in vzdrževanje}
\label{subsec:teorija_linux}

Linux je družina sistemov, ki delijo isto jedro, nad njim pa lahko teče različna programska oprema uporabniškega prostora.
Za dolgoročno vzdrževanje naprav je ključno, ali je podpora za strojno opremo vključena v glavno razvojno vejo jedra (angl.~\textit{mainline}) ali pa je vezana na proizvajalčevo prilagojeno različico jedra~\cite{kernelorg_releases}.
Če je podpora del glavne veje, lahko naprava praviloma dobiva varnostne popravke skupaj z ostalim ekosistemom.
Če pa je odvisna od zaprtih ali slabo vzdrževanih sprememb dobavitelja, je življenjska doba programske podpore praviloma krajša.

\subsection{Android: AOSP in ločitev dobaviteljskega sklada}
\label{subsec:teorija_android}

Android temelji na jedru Linux, vendar ima specifičen uporabniški prostor in nabor sistemskih storitev, ki skupaj tvorijo platformo~\cite{aosp_get_started_2025}.
Za odprtost je pomembna meja med generičnim sistemom in strojno specifičnimi deli.
Projekt Treble je uvedel bolj jasno ločitev na particije in vmesnike (npr.~ločen \texttt{system} in \texttt{vendor}), s čimer sistem teoretično omogoča posodabljanje neodvisno od dobaviteljskega sklada~\cite{android_treble_vendor_system_split,android_hidl_versioning_treble}.
Kljub temu dobaviteljski del pogosto vsebuje zaprt nabor knjižnic, implementacije abstrakcijskega sloja strojne opreme (angl.~\textit{Hardware Abstraction Layer}, HAL), gonilnike in strojno programsko opremo, brez katerih naprava ne deluje.

\section{Gonilniki, strojna programska oprema in binarni dodatki}
\label{sec:teorija_firmware}

Pri telefonih se pogosto srečamo s situacijo, kjer je del kode odprtokoden, ključne funkcionalnosti pa so dobavljene v obliki binarnih dodatkov (angl.~\textit{binary blobs}).
Pomembno je ločiti med:
\begin{description}
  \item[Gonilniki] kodo, ki teče v jedru ali uporabniškem prostoru in omogoča komunikacijo s strojno opremo.
  \item[Strojno programsko opremo] kodo, ki teče na sami komponenti (npr.~Wi-Fi čip, digitalni signalni procesor ali modem) in jo sistem zgolj naloži ali upravlja.
\end{description}

Tudi če so gonilniki odprtokodni, je strojna programska oprema lahko popolnoma zaprta, brez možnosti revizije ali popravkov.
To je ena glavnih tehničnih ovir za telefon, kjer bi bil celoten programski sklad pregledljiv in zamenljiv.
Najbolj problematični so podsistemi z visokimi privilegiji ali neposrednim vplivom na komunikacijo (npr.~osnovnopasovni modem in brezžični vmesniki).
Raziskave kažejo, da se ranljivosti in napadi pogosto selijo prav v zaprt dobaviteljski sklad in strojno programsko opremo, kjer je preglednost manjša in so orodja za analizo omejena~\cite{hernandez_firmwire_ndss2022,davis_cellular_baseband_security_2012}.

\section{Odprtost, varnost in zasebnost}
\label{sec:teorija_varnost}

Odprta koda lahko prispeva k varnosti, ker omogoča neodvisen pregled in hitrejše odkrivanje napak.
Vendar odprtost sama po sebi ni garancija; varnost je predvsem rezultat procesa (revizije, testiranja, odzivnost na ranljivosti) in nadzora nad tem, kaj se na napravi dejansko izvaja.
Pri telefonih zato poleg licence ocenjujemo tudi dobavno verigo, mehanizme posodabljanja ter stopnjo zaprtih komponent.

\subsection{Stranska vrata in dobavna veriga}
\label{subsec:teorija_backdoor}

Stranska vrata razumemo kot namerno dodano ali prikrito funkcionalnost, ki omogoča dostop mimo običajnih varnostnih mehanizmov.
Pri zaprtih komponentah je problem očiten: uporabnik in neodvisni raziskovalci težko preverijo, ali takšna funkcionalnost obstaja.
Tudi pri odprtokodni programski opremi pa ostaja pomemben napadni vektor dobavna veriga in orodja za izgradnjo.
Klasičen primer je Thompsonov opis, kako lahko kompromitiran prevajalnik vstavi stranska vrata v program tudi takrat, ko je izvorna koda navidezno \enquote{čista}~\cite{thompson_trusting_trust}.
Za oceno zaupanja je zato smiselno gledati širše od same izvorne kode.

Pri telefonih se pomisleki o stranskih vratih najpogosteje vežejo na:
\begin{itemize}
  \item zaprt dobaviteljski sklad na komunikacijskih čipih (modem, Wi-Fi/Bluetooth),
  \item privilegirane varnostne podsisteme (npr.~TEE in varnostni element) in
  \item lastniške sistemske storitve s širokimi dovoljenji (npr.~telemetrija).
\end{itemize}

\subsection{Integriteta zagona in ponovljive gradnje}
\label{subsec:teorija_supply_chain}

Moderni telefoni pogosto uporabljajo mehanizme preverjenega zagona, kjer je zagonska veriga kriptografsko vezana na podpisane sistemske slike.
To izboljša zaščito pred zlonamernimi spremembami, hkrati pa lahko omeji lastnika naprave, če je zagonski nalagalnik zaklenjen in proizvajalec ne omogoča nalaganja alternativnih podpisov~\cite{android_verified_boot}.
Za sisteme, ki ciljajo na odprtost, je zato pomembno ravnotežje: želimo visoko integriteto, vendar tudi možnost, da lastnik zakonito naloži in podpiše lastno zgradbo sistema.

Dodaten sloj zaupanja predstavljajo ponovljive gradnje (angl.~\textit{reproducible builds}), kjer lahko neodvisna stran iz iste izvorne kode izdela identične binarne pakete~\cite{reproducible_builds}.
S tem zmanjšamo tveganje, da bi bila uradna izdaja kompromitirana na poti od kode do binarne slike.
Pri telefonih je to posebej relevantno, ker uporabniki praviloma nameščajo vnaprej zgrajene sistemske slike, ne pa da bi sistem prevajali sami.

\subsection{Zasebnost kot posledica arhitekture}
\label{subsec:teorija_zasebnost}

Zasebnost pri telefonih ni samo vprašanje aplikacij, ampak tudi sistemske arhitekture.
Če naprava temelji na zaprtih storitvah in naborih API-jev, uporabnik težko oceni, kateri podatki se pošiljajo in kdaj.
Alternativni sistemi lahko zmanjšajo odvisnost od centraliziranih storitev, vendar se pri tem pogosto pojavi kompromis pri združljivosti aplikacij, zlasti pri aplikacijah z višjimi varnostnimi zahtevami, ki uporabljajo mehanizme preverjanja integritete~\cite{play_integrity_overview}.

\section{Vzdrževanje in življenjski cikel naprav}
\label{sec:teorija_vzdrzevanje}

Pri oceni odprtokodnega telefona je praktična uporabnost močno povezana z vzdrževanjem.
Če naprava ne dobiva varnostnih popravkov, odprtost hitro postane zgolj formalna lastnost.
Analize Android ekosistema kažejo velike razlike v dolžini podpore in zamudah pri varnostnih posodobitvah med proizvajalci~\cite{acar_android_security_updates_2024}.
V mobilnem Linuxu se ta problem pogosto rešuje z uporabo glavne razvojne veje jedra in standardnega uporabniškega prostora, vendar je cena praviloma slabša strojna podpora in manj zrel ekosistem aplikacij.

V naslednjih poglavjih bomo na podlagi teh izhodišč postavili merljive kriterije, kaj v našem kontekstu pomeni \enquote{popolnoma odprtokoden telefon}, nato pa bomo z njimi analizirali izbrane obstoječe rešitve.



%\cleardoublepage
%\addcontentsline{toc}{chapter}{Literatura}

\printbibliography[heading=bibintoc,type=article,title={Članki v revijah}]

\printbibliography[heading=bibintoc,type=inproceedings,title={Članki v zbornikih}]

\printbibliography[heading=bibintoc,type=incollection,title={Poglavja v knjigah}]

\printbibliography[heading=bibintoc,title={Celotna literatura}]

\end{document}

